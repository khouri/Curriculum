\section{Experiência com ensino superior}
\cventry{2018}{Aulas SENAC Santo Amaro (TCC II)}{\textsc{USP}}{São Paulo}{}{Ministrei aulas de TCC II no Senac Santo Amaro, foram 16h de aulas, com revisão da estrutura de trabalho acadêmico e desenvolvimento do trabalho durante a aula \newline{}}

\cventry{2018}{Aulas SENAC Santo Amaro (Técnicas de programação)}{\textsc{USP}}{São Paulo}{}{Ministrei aulas de técnicas de programação no Senac Santo Amaro, foram 48h de aulas, com trabalhos práticos em todas as aulas e uma prova final. \newline{}}

\cventry{2018}{Palestra sobre Machine learning em Cuzco}{\textsc{Peru}}{Cuzco}{}{Realizei uma palestra sobre machine learning em Cuxco por convite da universidade de federal de Cuzco com o tema de machine learning (\href{Link da palestra}{\url{https://github.com/khouri/Apresentacao_Cusco}}) \newline{}}

\cventry{2013-2016}{Aulas particulares de algoritmos e estruturas de dados (AED)}{\textsc{USP}}{São Paulo}{}{Fui professor particular de \(7\) alunos de mestrado, que apresentavam dificuldades na disciplina de algoritmos. Nessas aulas particulares localizei as dificuldades individuais dos alunos e adaptei minha metodologia de ensino em cada caso. O foco dessas aulas eram explicar as estruturas de dados clássicas: i) lista; ii)pilha; iii)deque; iv) árvores, em seguida o cálculo das complexidades assintóticas dessas estruturas. \newline{}}

\cventry{2014-2015}{ Monitoria na disciplina de AED}{\textsc{assistente de professor}}{São Paulo}{}{Fui assistente do professor Dr. Ivandré Paraboni na disciplina de AED ministrada na graduação durante um ano. As atividades iniciais da monitoria consistiram em participar da elaboração de provas, trabalhos de programação (usando a linguagem C) e criação da bateria de testes automatizados para corrigir os trabalhos. As atividades finais foram aplicação de provas aos alunos (o professor precisou viajar) e sanar dúvidas nos horários de monitoria. \newline{}}

\cventry{2014}{ Elaborei um curso de Latex}{\textsc{Plataforma Udemy}}{São Paulo}{}{Desenvolvi um curso de Latex com duração de \(30\) minutos na plataforma virtual Udemy (\href{https://www.udemy.com/grupo_latex/learn/v4/}{\url{https://www.udemy.com/grupo\_latex/learn/v4/}}). O curso contempla três sistemas operacionais (Win, Linux e Mac) e ensina alunos de graduação e/ou pós-graduação a desenvolver uma monografia em latex. \newline{}}

\cventry{2013-2014}{ Monitoria na disciplina de resolução de problemas}{\textsc{USP-assistente de professor}}{São Paulo}{}{Fui assistente dos professores Dr. Igari e Andréia nas disciplinas de resolução de problemas (durante um ano). Neste período orientei alunos a desenvolverem relatórios científicos (no modelo de iniciação científica). \newline{}}

\cventry{2013-2014}{ Coorientação de aluno de iniciação científica}{\textsc{USP-assistente de professor}}{São Paulo}{}{Meu orientador permitiu que eu orientasse um aluno de iniciação científica em conjunto. Orienteí o aluno a construir diversos scripts na plataforma R para analise de grandes quantidades de dados. \newline{}}


\section{Experiência no mercado de trabalho}
\cventry{Atual}{Cientista de Dados Sr.}{\textsc{NuvemShop}}{São Paulo}{}{Trabalho  como cientista de dados na empresa Nuvemshop onde aplico meu conhecimento para programar crawlers, criar modelos estatísticos e realizar relatórios que auxiliam a área de marketing. Desenvolvi um modelo de predição, em sete dias, de clientes pagadores (mais de 7 vendas em 90 dias) com F1 de $0.81$, atuei no levantamento de variáveis, engenharia de features, treinamento do modelo e deploy do modelo em produção. Atualmente estou trabalhando em outro preditor para clientes recorrentes. Trabalhei em Buenos Aires durante 1 mês para levantar requisitos com o cliente e realizar estudos exploratórios para o modelo citado.}
	
\cventry{2016--2018}{Cientista de Dados Sr.}{\textsc{PagSeguro}}{São Paulo}{}{Programei diversas aplicações Web Shiny: i) relatórios para mesa de análise; ii) relatórios para meu gestor; realizo manutenção da aplicação \emph{Command Center} usada pela equipe de risco para monitorar fraudes no sistema de risco.\newline{} 
Atuo como DBA do banco de dados colunar HP Vertica da equipe de risco realizando levantamento de espaço físico (HW), lógico (licença do Vertica), criando usuários, atribuindo roles e permissões, otimizando pesquisas SQL para equipe, matando processos travados no servidor. \newline{}
Soluciono problemas relativos a proveniência de dados para serem analisados, quando o ETL parou de funcionar e precisei localizar o erro dentro do PagSeguro, encontrar o responsável e ajudá-lo a implementar a solução adequada dentro do prazo de 6 horas (quando ocorreria o jogo do Brasil, esses dados eram nescessários para máquina de risco) foi nescessário atuar nas equipes de ETL, banco de dados e programação para solucionar o imprevisto da implantação da noite anterior.\newline{}
Realizo análise de vendedores com alta taxa de CBK ou baixa taxa de aprovação para tomar medidas adequadas que equilibrem o risco e a receita. Realizo estudos sobre otimização de regras da máquina de risco como o Totalizador, meu estudo mostrou que é possível remover até 25\% do totalizador com risco muito baixo para a saúde financeira da empresa. Elaborei uma comparação entre árvores de decisão (rpart e C50) para poder otimizar a redução do CBK da Cipsoft (parceiro Boa Compra) criando novas regras para bloqueio. Realizei uma clusterização por meio do algoritmo Kmeans com visão vendedor usando cinco variáveis. \newline{}
Soluciono problemas técnicos da equipe sobre computação e/ou programação em R usando meu background computacional avançado oriundo de minha época no Itaú-Unibanco. Oriento analistas Jr. formados em computação sobre como atuar em diversas situações desde incêndios (como perda de acesso ao banco de dados para a equipe toda, falha em ETL, etc.) até sobre arquitetura de sistemas, programação avançada (reflection e non standard evaluation) para garantir uma arquitetura desacoplada e escalável. Acompanho reuniões técnicas para melhorar a infra-estrutura da máquina de risco, com a HP para expor desafios e encontrar soluções de banco de dados em conjunto com consultores da HP, com a Microsoft e seus especialistas para instalar o deployR e a nuvem de machine learning que poderá ser usada na área de risco. \newline{}
Trabalhei também na criação de regras de risco para novos produtos da empresa como: i) P2P; ii) QR Code; iii) LK; onde interagi com a área de produtos para definir quais regras de risco podem ser usadas para barrar possíveis ataques. Refatorei a arquitetura da máquina de risco com objetivo de aumentar a organização, gerenciamento, escalabilidade das regras de risco (alteração do fluxo de informações ), para obter esse resultado interagi com a área de PeD da PagSeguro responsável pelo produto, com a Feedzai (empresa portuguesa responsável pela solução técnica) e programei a nova arquitetura de workflows.\newline{}
Atualmente estou mapeando, definindo e otimizando os processos da área de gestão de risco para permitir uma atuação mais eficiente da equipe e dos gestores.}

\cventry{2012--2014}{ Analista Jr.}{\textsc{Itaú Unibanco}}{São Paulo}{}{Atuação como analista desenvolvedor do sistema de marcação a mercado, accrual, resultado e risco (M6 calculadora). O sistema foi escrito em VBNET, \texttt{C\#}, a base de dados Sql Server 2008 e pacotes de SSIS 2005.
Atuando nas calculadoras, no processamento (geração de XML via código SQL) e integração de dados no sistema através de pacotes de SSIS 2005 (importação de dados oriundos dos sistemas boletadores).
\newline{}
Atuando também como analista funcional para especificar para diferentes fábricas de software: BRQ e Everis novos requisitos do sistema. Essas especificações continham diagramas de casos de uso, de sequência de sistema e descrições das funcionalidades.
Realizei diversas implantações com softwares desenvolvidos pela IBM: Clear Quest e Clear Case utilizados para catalogar pacotes em produção.
\newline{}
Por fim, atuação como analista de sustentação corrigindo bugs para o usuários em ambiente de produção. O principal projeto elaborado foi o projeto Colombia, que consistia no desenvolvimento de calculadoras de risco de mercado para a mesa de operações colombiana do Itaú. Esse projeto se destacou por apresentar um modelo de cálculo interamente novo para prever risco de mercado colombiano.
\newline{}}

%\cventry{2011--2012}{ Estagiário}{\textsc{Arbit}}{São Paulo}{}{Atuei como estagiário na empresa por 1 ano, desenvolvi softwares e soluções em base de dados. Participei de dois projetos, o primeiro foi um desenvolvimento de plugin para excel, um CRUD na base de dados para permitir gestores alocarem recursos em projetos. O plugin permitia diferentes permissões para cada grupo de usuários e permitia analisar o histórico de todas as alterações dos registros na base de dados.
%O segundo projeto onde participei foi o desenvolvimento de um CRM para uma empresa de venda de fármacos, a solução foi em plataforma .NET.
%\newline{}}
%
%\cventry{2010--2011}{ Estagiário}{\textsc{Spring Wirelles}}{São Paulo}{}{Atuei como estagiário na empresa Spring Wirelles  na área de testes automatizados, suporte, testes manuais, abertura de chamados e análise de chamados. As principais funções exercidas nesse estágio foram: consultar banco de dados, abrir chamados, reportar bugs do sistema, sugerir melhorias ao sistema, executar e analisar bateria de testes automatizados e liberar a versão do mSeries.
%\newline{}}
%
%\cventry{2009--2010}{ Estagiário}{\textsc{NEC}}{São Paulo}{}{Estágio na empresa nec brasil do ramo de hw e telefonia, atuando na área de suporte ao sw, atendendo chamados abertos pelo cliente dentro do sla estabelecido pelos contratos. Utilizando a tecnologia de vpn(para acessar o cliente), elaborando scripts com as tecnologias bash e scripts com consultas Oracle para corrigir incosistência de bases de dados.
%\newline{}}