\section{Formação}
\cventry{2017}{Doutorando em Ciências (Computação)}{Universidade de São Paulo}{SP}{\textit{Média--9,25}}{Atualmente estou cursando uma disciplina do curso de pós graduação no IME que aborda machine learning aplicada em biologia.}
	
\cventry{2016}{Mestre em Ciências (Computação)}{Universidade de São Paulo}{SP}{\textit{Média--9,25}}{Durante o mestrado trabalhei com diversas técnicas de inteligência artificial como: classificadores; regressores; e \emph{ensembles}, técnicas estatísticas para redução de dimensões do conjunto de dados como: PCA, ICA, PCA probabilístico e SVD. Essas técnicas foram usadas em um sistema de recomendação híbrido. Este sistema recomenda blocos de código fonte para biólogos que estão construíndo workflows científicos, o título da dissertação é: ``Desenvolvimento de técnica para recomendar atividades em workflows científicos: uma abordagem baseada em ontologias'' e se encontra no site de teses da USP ({\url{http://www.teses.usp.br/teses/disponiveis/100/100131/tde-19042016-140611/pt-br.php}}).}

\cventry{2008--2011}{Bacharel em Sistemas de Informação}{Universidade de São Paulo}{SP}{\textit{Média--7,2}}{TCC - Solução numérica e análise do comportamento assimptótico de equações diferenciais estocásticas.}